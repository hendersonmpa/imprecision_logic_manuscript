% Created 2022-04-14 Thu 17:25
% Intended LaTeX compiler: pdflatex
\documentclass[review]{elsarticle}
\usepackage[utf8]{inputenc}
\usepackage[T1]{fontenc}
\usepackage{graphicx}
\usepackage{longtable}
\usepackage{wrapfig}
\usepackage{rotating}
\usepackage[normalem]{ulem}
\usepackage{amsmath}
\usepackage{amssymb}
\usepackage{capt-of}
\usepackage{hyperref}
\usepackage{lineno}
\linenumbers
\usepackage{setspace}
\onehalfspacing
\authblk
\usepackage{pdfpages}
\usepackage{textpos}
\usepackage[final]{draftwatermark}
\usepackage{gensymb}
\usepackage{amsmath}
\usepackage{chemfig}
\setchemfig{atom style={scale=0.45}}
\usepackage[]{mhchem}
\date{}
\title{}
\hypersetup{
 pdfauthor={Matthew Henderson},
 pdftitle={},
 pdfkeywords={},
 pdfsubject={},
 pdfcreator={Emacs 27.2 (Org mode 9.5)}, 
 pdflang={English}}
\usepackage[notquote]{hanging}
\begin{document}\makeatletter
\newcommand{\citeprocitem}[2]{\hyper@linkstart{cite}{citeproc_bib_item_#1}#2\hyper@linkend}
\makeatother



\begin{frontmatter}
\title{Imprecision Logic}
\author[NSO, UoO]{Matthew P.A. Henderson\corref{cor1}}
\ead{mhenderson@cheo.on.ca}
\author[NSO]{Michael Kowalski}
\author[NSO, UO]{Pranesh Chakraborty}
\address[NSO]{Newborn Screening Ontario, Children's Hospital of Eastern Ontario,Canada}
\address[UoO]{Department of Medicine, University of Ottawa,Canada} 
\cortext[cor1]{Corresponding author}
\end{frontmatter}

\section*{Notes}
\label{sec:orga9f55b9}
\subsection*{Focused Report}
\label{sec:org1092d71}
\begin{itemize}
\item The Focused Report category is intended for concise method
evaluation contributions and succinct clinical manuscripts. All
Focused Reports will undergo peer review.
\item Submissions in this category should contain four sections:
\begin{itemize}
\item Abstract (structured, no more than 250 words)
\item Introduction
\item Methods
\item Results
\item Discussion
\item An Impact Statement should appear after the abstract.
\end{itemize}
\item They should be no more than 1,500 words in length with a maximum of
20 references and a total of no more than two tables and
figures. Figures and tables should not be multipart (i.e., Fig. 1A,
1B, 1C, Part 1, Part 2). No more than 5 authors should be
listed. Supplemental data are permitted for Focused Reports.
\end{itemize}

In some instances, editors may request that a submission of another article type to JALM be decreased to meet the requirements of a Focused Report.

\section*{{\bfseries\sffamily TODO} Abstract (250 words)}
\label{sec:org5d769c1}
\begin{description}
\item[{Introduction}] 

\item[{Methods}] 

\item[{Results}] 

\item[{Conclusion}] 
\end{description}
\section*{{\bfseries\sffamily TODO} Keywords}
\label{sec:org5bd1fea}
\section*{{\bfseries\sffamily TODO} Introduction}
\label{sec:orgd002189}
\begin{itemize}
\item The probability that a laboratory will incorrectly assign a
screen-positive or negative result owing to measurement error can be
estimated from the area under the standardized normal distribution.

\item The uncertainty of measurement approach uses an expansion factor
of 2. This would result an \(\sim\) 2\% probability of a false negative
result (Table \ref{tab:org3724d1a}).
\item The tolerance for a false negative first tier screening result at
NSO is very low, therefore, the most appropriate expansion factor
should be applied.

\item Analytical drift due to factors such as calibrations and weather can
result in periodic bias
\item This should also be considered in determining impression logic
\end{itemize}

\begin{table}[htbp]
\caption[sigma]{\label{tab:org3724d1a}Probability of a false negative screen due to imprecision}
\centering
\begin{tabular}{rrl}
SD & probability of false negative & count\\
\hline
1 & 0.1586553\\
2 & 0.02275013\\
3 & 0.001349898\\
4 & 3.167124ee-05\\
5 & 2.866516ee-07\\
6 & 9.865876ee-10\\
\end{tabular}
\end{table}


\begin{table}[htbp]
\caption[sigma]{\label{tab:orgea62171}Precison near screening thresholds}
\centering
\begin{tabular}{lrlrr}
Analyte & Threshold & QC & mean & sd\\
\hline
GALT & 1.5 & G17069 & 1.6 & 0.2\\
BIOT & 27 & B170621 & 33 & 3.7\\
TSH & 15 & 659845-1 & 15 & 1.3\\
\end{tabular}
\end{table}


\section*{{\bfseries\sffamily TODO} Material and Methods}
\label{sec:orgb6b5763}
\section*{Results}
\label{sec:org1eb909c}

% Table created by stargazer v.5.2.2 by Marek Hlavac, Harvard University. E-mail: hlavac at fas.harvard.edu
% Date and time: Thu, Apr 14, 2022 - 05:25:52 PM
\begin{table}[!htbp] \centering 
  \caption{Uncertainty of Measurement Based Initial Screening Thresholds with Predicted Repeat Samples Population Data} 
  \label{} 
\begin{tabular}{@{\extracolsep{5pt}} ccccc} 
\\[-1.8ex]\hline 
\hline \\[-1.8ex] 
 & factor & confirm & initial & grey\_samples \\ 
\hline \\[-1.8ex] 
1 & $2$ & $1.523$ & $1.888$ & $24.650$ \\ 
2 & $3$ & $1.523$ & $2.091$ & $52.200$ \\ 
3 & $4$ & $1.523$ & $2.294$ & $94.250$ \\ 
4 & $5$ & $1.523$ & $2.496$ & $156.600$ \\ 
5 & $6$ & $1.523$ & $2.699$ & $234.900$ \\ 
\hline \\[-1.8ex] 
\end{tabular} 
\end{table}


\begin{center}
\includegraphics[width=.9\linewidth]{./figures/galtthresholds.pdf}
\end{center}

\section*{{\bfseries\sffamily TODO} Discussion}
\label{sec:org60deb34}
\section*{{\bfseries\sffamily TODO} Conclusions}
\label{sec:org2571a69}

\section*{Acknowledgments}
\label{sec:org2e4df9b}
Funding: None.
\section*{References}
\label{sec:org4e6be2f}
\begin{hangparas}{1.5em}{1}

\end{hangparas}
\end{document}