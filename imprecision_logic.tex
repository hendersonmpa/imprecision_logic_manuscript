% Created 2022-04-26 Tue 16:58
% Intended LaTeX compiler: pdflatex
\documentclass[review]{elsarticle}
\usepackage[utf8]{inputenc}
\usepackage[T1]{fontenc}
\usepackage{graphicx}
\usepackage{longtable}
\usepackage{wrapfig}
\usepackage{rotating}
\usepackage[normalem]{ulem}
\usepackage{amsmath}
\usepackage{amssymb}
\usepackage{capt-of}
\usepackage{hyperref}
\usepackage{lineno}
\linenumbers
\usepackage{setspace}
\onehalfspacing
\authblk
\usepackage{pdfpages}
\usepackage{textpos}
\usepackage[final]{draftwatermark}
\usepackage{gensymb}
\usepackage{amsmath}
\usepackage{chemfig}
\setchemfig{atom style={scale=0.45}}
\usepackage[]{mhchem}
\date{}
\title{}
\hypersetup{
 pdfauthor={Matthew Henderson},
 pdftitle={},
 pdfkeywords={},
 pdfsubject={},
 pdfcreator={Emacs 27.2 (Org mode 9.5)}, 
 pdflang={English}}
\usepackage[notquote]{hanging}
\begin{document}\makeatletter
\newcommand{\citeprocitem}[2]{\hyper@linkstart{cite}{citeproc_bib_item_#1}#2\hyper@linkend}
\makeatother



\begin{frontmatter}
\title{Imprecision Logic}
\author[NSO, UoO]{Matthew P.A. Henderson\corref{cor1}}
\ead{mhenderson@cheo.on.ca}
\author[NSO]{Michael Kowalski}
\author[NSO, UO]{Pranesh Chakraborty}
\address[NSO]{Newborn Screening Ontario, Children's Hospital of Eastern Ontario,Canada}
\address[UoO]{Department of Medicine, University of Ottawa,Canada} 
\cortext[cor1]{Corresponding author}
\end{frontmatter}

\section*{Notes}
\label{sec:org1061138}
\subsection*{Focused Report}
\label{sec:org21b4f83}
\begin{itemize}
\item The Focused Report category is intended for concise method
evaluation contributions and succinct clinical manuscripts. All
Focused Reports will undergo peer review.
\item Submissions in this category should contain four sections:
\begin{itemize}
\item Abstract (structured, no more than 250 words)
\item Introduction
\item Methods
\item Results
\item Discussion
\item An Impact Statement should appear after the abstract.
\end{itemize}
\item They should be no more than 1,500 words in length with a maximum of
20 references and a total of no more than two tables and
figures. Figures and tables should not be multipart (i.e., Fig. 1A,
1B, 1C, Part 1, Part 2). No more than 5 authors should be
listed. Supplemental data are permitted for Focused Reports.
\end{itemize}

In some instances, editors may request that a submission of another article type to JALM be decreased to meet the requirements of a Focused Report.

\section*{{\bfseries\sffamily TODO} Abstract (250 words)}
\label{sec:orga72e1c6}
\begin{description}
\item[{Introduction}] 

\item[{Methods}] 

\item[{Results}] 

\item[{Conclusion}] 
\end{description}
\section*{{\bfseries\sffamily TODO} Keywords}
\label{sec:org2f9a7dc}
\section*{{\bfseries\sffamily STARTED} Introduction}
\label{sec:org2db0db8}

Newborn screening commonly uses algorithms based on biomarkers
measured in dried blood spot samples. These screening algorthims
generally involve two or more phases of testing. All samples are
tested in the first phase and during this phase sensitivity is
emphasised over specificity. The first phase identifies samples with
higher probability of disease for additional testing using an approach
that ideally maintains high sensitivity with improved
specificity. Ideally an orthogonal analytical method or a different
biomarker is examined in the second phase of screening, however in a
number of disorders the same method is used in the second phase of
screening to measure the sample in duplicate with a more specific
screening threshold applied to the mean of biomarker measurements
(Figure \ref{fig:org2033808}). This approach improves the precision of the
final screening result to reduce the change that a sample will be
deemed screen negative due to analytical imprecision.

Screen positive thresholds in newborn screening are set based on the
distribution of the biomarker: in the healthy population, the affected
population and specific sub-populations based on demographics such as
age, gestational age and birth weight. The goal of the laboratory is
then to identify samples with abnormal biomarker results. The
uncertainty of measurement of the screening method for analyte
concentrations near the screening threshold is useful in determining
an appropriate initial screen threshold to prompt repeat testing and
reduce the effect analytical imprecision on screening
determinations. The probability that a laboratory will incorrectly
assign a screen positive or negative determination due to measurement
error can be estimated from the area under the standardized normal
distribution.

Newborn screening for galactosemia at Newborn Screening Ontario is one
of a number of disease that follow the screening algorithm described
above. Newborn screening for galactosemia is based on measurement of
galactose-1-phosphate uridylyltransferase (GALT) activity in dried
blood spots. All samples are testing for GALT activity, GALT activity
is then tested in duplicate for all samples with activity below the
initial screening threshold. The mean of the three results is then
compared to the confirmation threshold. All results of the
confirmation phase are also subjected to medical scientific
review. Newborn screening for galactosemia will be used to demonstrate
the use of uncertainty of measurement for establishing thresholds in a
multi-stage testing algorithm, however this approach is applicable to
many testing scenarios in which biomarker results in a "grey-area"
must be identified for further investigation.

\begin{itemize}
\item Analytical drift due to factors such as calibrations and weather can
result in periodic bias
\item This should also be considered in determining impression logic
\end{itemize}



\begin{figure}[htbp]
\centering
\includegraphics[width=0.9\textwidth]{./figures/screening_phases.pdf}
\caption{\label{fig:org2033808}Simplified Screening Algorithm}
\end{figure}


\section*{{\bfseries\sffamily TODO} Material and Methods}
\label{sec:orgca84962}
\subsection*{GALT Activity Data}
\label{sec:orgeb0fc9d}
The laboratory information system was queried for all newborn
screening biotinidase activity results from the first phase of
screening in the period 2017-01-01 to 2018-12-31 (n = 285,875). This
data was used to determine the probability density function for GALT
activity in the population.

During the study period dried blood spot GALT activity were measured
on the SpotCheck Pro platform (Astoria Pacific, Oregon USA.). GALT
activity was reported in units per gram of hemoglobin (U/g Hb). 903
filter paper was used for dried blood spot sample collection (EBF,
South Carolina, USA).



\subsection*{{\bfseries\sffamily TODO} Uncertainty of Measurement}
\label{sec:org9f09060}
The precision of the GALT activity assay at the confirm screening
threshold (1.5 U/g Hg) was determined based on six months of quality
control (QC) data for a QC material with mean GALT activity of 1.6 U/g Hg.
\textbf{How many data points in six months of QC data}

\subsection*{Simulation}
\label{sec:orgeb18b94}


The probability that a sample will will be incorrectly assigned a
screen positive or negative result owing to measurement error was
estimated from the area under the standardized normal distribution
with mean (\(\mu\)) equal to the confirmation threshold, and the standard
deviation (\(\theta\)) of quality control results near the confirmation
threshold (Equation \ref{eq:error}).

\begin{equation}\label{eq:error}
X \sim N(\mu, \theta^2)
\end{equation}

The number of samples potentially affected by measurement imprecision
at the screening threshold can be estimated by the area under the
probability density function of measurement error after scaling for
the probability of a result at that concentration in the population.

\begin{equation}\label{eq:scaled}
 N(\mu, \theta^2) \cdot Pr_population[GALT = 1.5] 
\end{equation}


This probability distribution for measurement error was superimposed
on the distribution of GALT activity results from the population in
order to determine the number of samples that are potentially affected
by measurement imprecision at the screening threshold (Figure
\ref{fig:org34c08a4}).


\begin{itemize}
\item The uncertainty of measurement approach uses an expansion factor
of 2. This would result an \(\sim\) 2\% probability of a false negative
result (Table \ref{tab:orga20f5eb}).
\item The tolerance for a false negative first tier screening result at
NSO is very low, therefore, the most appropriate expansion factor
should be applied.
\end{itemize}




\subsection*{{\bfseries\sffamily TODO} Software}
\label{sec:org726e3e8}
The manuscript was prepared using the Org-mode environment for
literate programming and reproducible research (\hyperlink{citeproc_bib_item_1}{Schulte et al. 2012}). The R language for statistical computing was used
for all data analysis with tidyverse packages for data manipulation,
\textbf{update} (\hyperlink{citeproc_bib_item_2}{Team 2020}; \hyperlink{citeproc_bib_item_3}{Wickham et al. 2019}). R scripts used for data
analysis are available here:
\url{https://github.com/hendersonmpa/imprecision\_logic\_manuscript.git}



\section*{{\bfseries\sffamily STARTED} Results}
\label{sec:org750c59d}

A simulation was run to determine the number of samples in the "grey
area" and "uncertain area" as a function of the GALT activity
threshold used in the initial phase of screening (Figure
\ref{fig:org34c08a4}). The grey area is the region between the initial and
confirmation thresholds and all samples with GALT activity in this
range are repeated (Figure \ref{fig:org2033808}). The uncertain area is based
on the analytical imprecision of the GALT assay at the confirmation
threshold.

The scaled probability distribution for measurement error was
superimposed on the distribution of GALT activity results from the
population in order to determine the number of samples that are
potentially affected by measurement imprecision at the screening
threshold 

The number of samples in the grey area is determined by the distance
between the initial and confirmation thresholds and can be estimated
by the area in this region of the probability density function of GALT
activity in the population.
% latex table generated in R 4.0.3 by xtable 1.8-4 package
% Tue Apr 26 16:58:08 2022
\begin{table}[ht]
\centering
\begin{tabular}{rrrr}
  \hline
Expansion Factor & Initial Positive Threshold & Grey Area Samples & Analytically Uncertain Samples \\ 
  \hline
  0 & 1.50 & 0.00 & 9.5 \\ 
    1 & 1.70 & 10.15 & 3.1 \\ 
    2 & 1.90 & 23.20 & 0.45 \\ 
    3 & 2.10 & 49.30 & 0.027 \\ 
    4 & 2.30 & 92.80 & 0.00066 \\ 
    5 & 2.50 & 142.10 & 6e-06 \\ 
    6 & 2.70 & 221.85 & 9.2e-23 \\ 
   \hline
\end{tabular}
\caption{Uncertainty of Measurement Based Initial Screening Thresholds with Predicted Repeat Samples Population Data} 
\label{tab:uncertain}
\end{table}


\begin{figure}[htbp]
\centering
\includegraphics[width=0.9\textwidth]{./figures/galtthresholds.pdf}
\caption{\label{fig:org34c08a4}The left side of the GALT activity population distribution.}
\end{figure}

\section*{{\bfseries\sffamily TODO} Discussion}
\label{sec:org7e2fc84}
\begin{table}[htbp]
\caption[sigma]{\label{tab:orga20f5eb}Probability of a false negative screen due to imprecision}
\centering
\begin{tabular}{rrl}
SD & probability of false negative & count\\
\hline
1 & 0.1586553\\
2 & 0.02275013\\
3 & 0.001349898\\
4 & 3.167124ee-05\\
5 & 2.866516ee-07\\
6 & 9.865876ee-10\\
\end{tabular}
\end{table}


\section*{{\bfseries\sffamily TODO} Conclusions}
\label{sec:orgfbe9f78}

\section*{{\bfseries\sffamily DONE} Acknowledgments}
\label{sec:org3cb0f41}
Funding: None.
\section*{References}
\label{sec:org7da03af}
\begin{hangparas}{1.5em}{1}
\hypertarget{citeproc_bib_item_1}{Schulte, Eric, Dan Davison, Thomas Dye, and Carsten Dominik. 2012. “A Multi-Language Computing Environment for Literate Programming and Reproducible Research.” \textit{Journal of Statistical Software} 46 (3):1–11. \href{https://doi.org/10.18637/jss.v046.i03}{https://doi.org/10.18637/jss.v046.i03}.}

\hypertarget{citeproc_bib_item_2}{Team, R Core. 2020. \textit{R: A Language and Environment for Statistical Computing}. Vienna, Austria: R Foundation for Statistical Computing. \href{https://www.R-project.org/}{https://www.R-project.org/}.}

\hypertarget{citeproc_bib_item_3}{Wickham, Hadley, Mara Averick, Jennifer Bryan, Winston Chang, Lucy D’Agostino McGowan, Romain François, Garrett Grolemund, et al. 2019. “Welcome to the tidyverse.” \textit{Journal of Open Source Software} 4 (43):1686. \href{https://doi.org/10.21105/joss.01686}{https://doi.org/10.21105/joss.01686}.}
\end{hangparas}
\end{document}